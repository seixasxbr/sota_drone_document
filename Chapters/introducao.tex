\chapter{Introdução}
\label{chap:intro}

Este documento aborda o estudo do estado da arte de veículos aéreos não tripulados (VANTs) do tipo quadrotor, apresentando os principais estudos acadêmicos, técnicas e modelos dos últimos anos para embasar o desenvolvimento de um projeto envolvendo a concepção de um drone desse tipo. 



% Este pode ser um parágrafo citado por alguém \cite{Barabasi2003-1} e \cite{barabasi2003linked}.
% Para ajustar veja o comentário do capítulo \ref{chap:fundteor}.

% As orientações do robô são em três dimensões \cite{aperea-1}.
% Segundo \citeonline{aperea-1}

% Segundo \citeonline{barabasi2003linked}, ...

%--------- NEW SECTION ----------------------
\section{Objetivos}
\label{sec:obj}

% \begin{justifY}
%   Este projeto consiste em desenvolver um robô bípede de pequeno porte, ou seja que se desloca sobre dois pés. O robô deve ser capaz de se locomover e desviar de obstáculos em um determinado ambiente.
% \end{justify}

Este estudo foi realizado para dar suporte no denvolvimento de um quadrotor autônomo. Com objetivo de trazer conhecimento das melhores técnicas que vem sendo utilizadas em áreas como navegação, controle e localização e mapeamento simultâneos (SLAM), assim como os melhores modelos e arquiteturas para conceber um veículo eficiente.
\label{sec:obj}

%//todo incluir justify e p flushright 

% \subsection{Objetivos Específicos}
% \label{ssec:objesp}
% Os objetivos específicos deste projeto são:
% \begin{itemize}
%       \item Desenvolver habilidades de gestão de projetos.
%       \item Desenvolver algoritmos utilizando ROS;
%       \item Utilizar visão computacional;
%       \item Simular um robô no gazebo;
%   \end{itemize}

% \subsubsection*{Objetivos específicos principais}
% \label{sssec:obj-principais}
% ok vendo Aqui


% \begin{equation}
% \label{eq:energia}
%   E=mc
% \end{equation}


% \begin{equation*}
%   m=(\frac{E}{c})
% \end{equation*}


% \begin{equation*}
%   m=\Bigg(\frac{E}{c}\Bigg)
% \end{equation*}


% \begin{equation}
%   m=E/c
% \end{equation}


%--------- NEW SECTION ----------------------
\section{Justificativa}
\label{sec:justi}

Os VANTs tem sido cada vez mais utilizados no dia a dia das pessoas. Tarefas que envolvem risco podem ser facilmente executadas por esse tipo de aeronave sem expor o piloto aos perigos associados a essa missão. Os drones tem sido utilizados em áreas como cinematografia, cartografia, vigilância, entrega de encomendas, mapeamento, entre outras. Devido a isso surge a importância de estudar essas aeronaves que apresentam alguns desafios a serem enfrentados como autonomia de vôo, localização e controle.




%--------- NEW SECTION ----------------------
\section{Organização do documento}
\label{section:organizacao}

Este documento apresenta $5$ capítulos e está estruturado da seguinte forma:

\begin{itemize}

  \item \textbf{Capítulo \ref{chap:intro} - Introdução}: Contextualiza o âmbito, no qual a pesquisa proposta está inserida. Apresenta, portanto, a definição do problema, objetivos e justificativas da pesquisa e como este \thetypeworkthree está estruturado;
  \item \textbf{Capítulo \ref{chap:fundteor} - Fundamentação Teórica}: XXX;
  \item \textbf{Capítulo \ref{chap:metod} - Materiais e Métodos}: XXX;
  \item \textbf{Capítulo \ref{chap:result} - Resultados}: XXX;
  \item \textbf{Capítulo \ref{chap:conc} - Conclusão}: Apresenta as conclusóes, contribuições e algumas sugestões de atividades de pesquisa a serem desenvolvidas no futuro.

\end{itemize}
